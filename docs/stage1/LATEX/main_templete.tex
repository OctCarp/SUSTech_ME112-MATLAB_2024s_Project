\documentclass[10.5pt,compsoc,UTF8]{CjC}
\usepackage{CTEX}
\usepackage{graphicx}
\usepackage{footmisc}
\usepackage{subfigure}
\usepackage{url}
\usepackage{multirow}
\usepackage{multicol}
\usepackage[noadjust]{cite}
\usepackage{amsmath,amsthm}
\usepackage{amssymb,amsfonts}
\usepackage{booktabs}
\usepackage{color}
\usepackage{ccaption}
\usepackage{booktabs}
\usepackage{float}
\usepackage{fancyhdr}
\usepackage{caption}
\usepackage{xcolor,stfloats}
\usepackage{comment}
\setcounter{page}{1}
\graphicspath{{figures/}}
\usepackage{cuted}%flushend,
\usepackage{captionhack}
\usepackage{epstopdf}
\usepackage{gbt7714}
\usepackage{geometry}


%===============================%

%footnote use of *
\renewcommand{\thefootnote}{\fnsymbol{footnote}}
\setcounter{footnote}{0}
\renewcommand\footnotelayout{\zihao{5-}}

\newtheoremstyle{mystyle}{0pt}{0pt}{\normalfont}{1em}{\bf}{}{1em}{}
\theoremstyle{mystyle}
\renewcommand\figurename{figure~}
\renewcommand{\thesubfigure}{(\alph{subfigure})}
\newcommand{\upcite}[1]{\textsuperscript{\cite{#1}}}
\renewcommand{\labelenumi}{(\arabic{enumi})}
\newcommand{\tabincell}[2]{\begin{tabular}{@{}#1@{}}#2\end{tabular}}
\newcommand{\abc}{\color{white}\vrule width 2pt}
\renewcommand{\bibsection}{}
\makeatletter
\renewcommand{\@biblabel}[1]{[#1]\hfill}
\makeatother
\setlength\parindent{2em}

%调整页边距
\geometry{top=3cm, bottom=2cm, left=2cm, right=2cm}

\begin{document}

\makeatletter
\newcommand\mysmall{\@setfontsize\mysmall{7}{9.5}}
\newenvironment{tablehere}
  {\def\@captype{table}}

\let\temp\footnote
\renewcommand \footnote[1]{\temp{\zihao{-5}#1}}


{
\centering
{\zihao{2} \heiti MATLAB Project 选题报告 }
\vskip 5mm
{\zihao{3} \heiti ——基于 MATLAB 的面部表情识别 }
\vskip 5mm
}
{
\centering
{\zihao{4}\fangsong 郭健阳 12111506\quad  徐春晖 12110304}
\vskip 5mm
}
\vspace {10mm}

\begin{multicols}{2}
%%%%%%%%%%%%%%%%%%%%%%%%%%%%%%%%%%%%%%%%%%
%%%%%%%%%%%%%%%%%%%%%%%%%%%%%%%%%%%%%%%%%%

\section{\heiti 项目主题}

基于 MATLAB 的图像识别中的面部表情识别。通过输入的人脸表情图片,来分析其作为各个不同表情的可能性。

本项目与深度学习、神经网络、计算机视觉方向较为相关。着眼于 MATLAB 高效科学计算的优势,实现一个简单的面部表情识别。

%%%%%%%%%%%%%%%%%%%%%%%%%%%%%%%%%%%%%%%%%%
%%%%%%%%%%%%%%%%%%%%%%%%%%%%%%%%%%%%%%%%%%

\section{\heiti 项目目标}

使用 MATLAB 实现图像识别中的面部表情识别功能。具体来说,提供人物图片,识别图片中人物的面部表情,并给出预测表情的概率。

实现 GUI 界面,方便用户进行更好的交互。

因为本小组的两位同学都来自于计算机系,对于更普遍的科学理论,如生物、物理、电子等方面的背景知识了解有限,故决定:选取与日常学习内容有一定关系的计算机视觉方向,作为 MATLAB 课程项目的选题。

%%%%%%%%%%%%%%%%%%%%%%%%%%%%%%%%%%%%%%%%%%
%%%%%%%%%%%%%%%%%%%%%%%%%%%%%%%%%%%%%%%%%%

\section{\heiti 初步方案}

项目暂定分为以下几个行动步骤:

\begin{enumerate}
    \item 调研相关领域的研究成果和实现,收集有关面部表情识别的相关工作。
    \item 找到合适的训练数据集和深度学习架构,能合适地完成人脸表情识别的任务。
    \item 在 MATLAB 中使用合适的库实现深度学习网络结构。
    \item 训练模型,分析模型的准确度;使模型能够持久化保存。
    \item 设计 GUI 界面,让用户能够更直观地进行导入数据(即表情图片)、分析数据的操作。
\end{enumerate}

%%%%%%%%%%%%%%%%%%%%%%%%%%%%%%%%%%%%%%%%%%
%%%%%%%%%%%%%%%%%%%%%%%%%%%%%%%%%%%%%%%%%%

\section{\heiti 成员分工}
\begin{itemize}
    \item 郭健阳
        \begin{itemize}
            \item 理论资料收集;
            \item 数据集收集;
            \item GUI 设计;
        \end{itemize}
    \item 徐春晖
        \begin{itemize}
            \item 模型设计与实现;
            \item 性能测试;
        \end{itemize}
\end{itemize}

%%%%%%%%%%%%%%%%%%%%%%%%%%%%%%%%%%%%%%%%%%
%%%%%%%%%%%%%%%%%%%%%%%%%%%%%%%%%%%%%%%%%%

\vspace {10mm}

\centerline
{\zihao{5}\textsf{参~考~资~料}}
\zihao{5-} \addtolength{\itemsep}{-1em}
\vspace {1.5mm}

% \begin{thebibliography}{00}
% \bibitem{b1} Zhang, T., Zheng, W., Cui, Z., Zong, Y., Yan, J., \& Yan, K. (2016). A deep neural network-driven feature learning method for multi-view facial expression recognition. IEEE Transactions on Multimedia, 18(12), 2528-2536.
% \bibitem{b2} Mollahosseini, A., Chan, D., \& Mahoor, M. H. (2016, March). Going deeper in facial expression recognition using deep neural networks. In 2016 IEEE Winter conference on applications of computer vision (WACV) (pp. 1-10). IEEE.
% \bibitem{b3} Yang, H., Ciftci, U., \& Yin, L. (2018). Facial expression recognition by de-expression residue learning. In Proceedings of the IEEE conference on computer vision and pattern recognition (pp. 2168-2177).
% \end{thebibliography}

\bibliographystyle{plain}
\bibliography{references}
\nocite{*}

\end{multicols}
\end{document}


